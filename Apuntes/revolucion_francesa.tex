\subsection{Revolución Francesa}

\begin{itemize}

\item Causas políticas:
\begin{enumerate}
	\item Las costosas guerras de los reyes franceses (1756 - 1768).
\end{enumerate}

\item Causas económicas:
\begin{enumerate}
	\item Surgimiento de la clase media (burguesía).
	\item La oposición de la clase media al mercantilismo.
	\begin{enumerate}
		\item Liberalismo económico de Adam Smith (el estado no debe inmiscuirse en el mercado).
		\item John Keynes (``A largo plazo todos estaremos muertos'').
	\end{enumerate}
\end{enumerate}

\item Causas intelectuales.
\begin{enumerate}
	\item Teoría liberal de John Locke (división de poderes del estado, ``Dos Ensayos sobre el Poder Civil'').
	\item Teoría democrática de Rousseau (inspirador del Dr. Francia).
\end{enumerate}

\end{itemize}

\subsubsection{Napoleón Bonaparte}

\begin{itemize}

\item Puso fin al periodo revolucionario.
\item Transmitió los ideales de la revolución por toda Europa.

\end{itemize}