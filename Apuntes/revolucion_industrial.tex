\subsection{Revolución Industrial}

\begin{itemize}

\item Hacia finales del siglo XVIII, se revolucionó la industria textil en Inglaterra (\emph{primera revolución}).

\item La segunda revolución se produjo entre 1880 y 1890.
\begin{enumerate}
	\item Durante esta época las clases medias buscaron el derecho a votar, a la huelga, etc.
	\item Ante los conflictos sociales surgieron:
	\begin{itemize}
		\item El anarquismo.
		\item El socialismo utópico.
		\item El socialismo científico (marxismo).
	\end{itemize}
	\item La iglesia sentó su postura con la encíclica \emph{Rerum Novarum} (León XIII).
\end{enumerate}

\item Adelantos técnicos:
\begin{enumerate}
	\item Telégrafo.
	\item Teléfono.
	\item Pararrayos.
	\item Pila.
	\item Máquinas de vapor.
\end{enumerate}

\item Contribuciones:
\begin{enumerate}
	\item Louis Pasteur: esterilización.
	\item Charles Darwin: ``Origen de las Especies''.
	\item Augusto Comte: positivismo.
	\item Charles Dickens: ``Tiempos Difíciles''.
	\item Emile Zola: ``Yo Acuso''.
	\item Walt Whitman: ``Hojas de Hierba''.
	\item Vincent van Gogh: ``Girasoles''.
\end{enumerate}

\item La revolución industrial dividió al mundo en:
\begin{enumerate}
	\item Países productores de materias primas.
	\item Países industrializados.
\end{enumerate}

\end{itemize}