\subsection{Antigua Grecia}

\begin{itemize}

\item Las primeras informaciones encontramos en:
\begin{enumerate}
	\item ``La Ilíada'' (de Homero).
	\item ``La Odisea'' (de Homero).
	\item ``Los Trabajos y los Días'' (de Hesiodo).
	\item ``Teogonía'' (de Hesiodo).
\end{enumerate}

\item La guerra de Troya terminó en 1.200 A.C.

\item En el siglo VIII A.C. surgieron las \emph{polis} (ciudades estado). Las más importantes fueron Esparta y Atenas.

\item En el siglo V A.C. los persas invadieron Grecia, pero fueron derrotados.

\item Tras la guerra, Atenas se convirtió en la polis hegemónica, hecho que la enfrentó con Esparta (guerra del Peloponeso, 460 A.C. a 404 A.C.).

\item Atenas fue derrotada, pero la anarquía se instaló en la región.

\item Filipo, rey macedonio, ocupó Grecia. Su hijo Alejandro Magno lidero a los griegos contra los persas (se inicia la helenización del mundo antiguo).

\item Alejandro falleció en el 323 A.C. Su m uerte puso fin a la civilización helenística.

\item Cultura:

\begin{itemize}

	\item Filosofía:
	\begin{enumerate}
		\item Thales de Mileto:
		\begin{itemize}
			\item Fundador o padre de la filosofía.
			\item Concepto del \emph{arque}, origen de todas las cosas (el agua).
		\end{itemize}
		\item Pitagoras.
		\item Sócrates.
		\item Platón.
		\item Aristóteles.
	\end{enumerate}
	
	\item Historia:
	\begin{enumerate}
		\item Herodoto: ``Los Nueve Libros de la Historia''.
		\item Tucídices.
		\item Jenofonte.
	\end{enumerate}
	
	\item Medicina:
	\begin{enumerate}
		\item Aristóteles.
		\item Hipócrates.
	\end{enumerate}
	
	\item Matemáticas:
	\begin{enumerate}
		\item Pitágoras.
		\item Euclides.
		\item Arquímedes.
	\end{enumerate}
	
	\item Astronomía:
	\begin{enumerate}
		\item Aristarco: La Tierra gira alrededor del Sol.
		\item Ptolomeo: La Tierra es plana.
		\item Eratóstenes: La Tierra es redonda.
	\end{enumerate}
	
	\item Literatura:
	\begin{enumerate}
		\item Sófocles: ``Edipo Rey'', ``Antígona''.
		\item Esquilo: ``Los Persas'', ``Prometeo Encadenado''.
		\item Eurípides: ``Las Troyanas''.
	\end{enumerate}

\end{itemize}

\item Democracia de la Antigua Grecia:
\begin{itemize}
	\item Solo para los varones.
	\item Directa.
\end{itemize}

\item Gobierno de:
\begin{itemize}
	\item Esparta:
		\begin{itemize}
			\item Oligarquía militar conservadora.
			\item A los niños se les enseñaba gimnasia, poesía y música.
		\end{itemize}
	\item Atenas:
		\begin{itemize}
			\item Democracia directa (la democracia moderna es representativa e indirecta).
		\end{itemize}
\end{itemize}

\end{itemize}