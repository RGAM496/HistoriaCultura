\subsection{Renacimiento}

\begin{itemize}

\item Origen: Italia.

\item Significa volver a nacer o renacer de las colturas clásicas.
No es imitación, pues incorpora elementos nuevos.

\item Precursores:
\begin{enumerate}
	\item Dante Alighieri (``La Divina Comedia'').
	\item Giovanni Bocaccio.
	\item Francesco Petrarca.
	\item Leonardo Da Vinci (``La Última Cena'', ``Monalisa'').
	\item Miguel Ángel (pintó la Capilla Sixtina).
	\item Rafael (``Moisés'', ``David'', ``La Piedad'').
	\item Nicolás Maquiavelo (``El Príncipe'').
	\item William Shakespeare (``Hamlet'').
	\item Miguel de Cervantes (``El Ingenioso Hidalgo Don Quijote De La Mancha'').
\end{enumerate}

\item Astronomía:
\begin{enumerate}
	\item Nicolás Copérnico (los planetas giran alrededor del Sol).
	\item Galileo Galilei (comprueba la teoría de Copérnico).
\end{enumerate}

\item Medicina:
\begin{enumerate}
	\item Andreas Vesalius (Atlas del cuerpo humano).
	\item Nicolás Servet (circulación de la sangre).
\end{enumerate}

\item Otros descubrimientos:
\begin{enumerate}
	\item Pólvora.
	\item Caravela.
	\item Reloj mecánico.
	\item Brújula.
	\item Imprenta.
\end{enumerate}

\item El hombre del renacimiento busca disfrutar de la vida, enriquecerse.

\item El \emph{método deductivo} (de lo general a lo particular) de Aristóteles es reemplazado por el \emph{método inductivo} (de lo particular a lo general) de Rogelio Bacón.

\end{itemize}