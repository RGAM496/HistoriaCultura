\subsection{Roma}

\begin{itemize}

\item Fundación: 753 A.C.

\item Primitivos habitantes:
\begin{enumerate}
	\item Etruscos.
	\item Latinos.
	\item Griegos.
\end{enumerate}

\item Periodos de su historia:
\begin{enumerate}
	\item Monarquía: 753 A.C. a 509 A.C.
	\item República: 509 A.C. a 27 A.C.
	\item Imperio: 27 A.C. a 476 D.C. (Occidente) y 1.453 D.C. (Oriente).
\end{enumerate}

\end{itemize}

\subsubsection{La Monarquía}

\begin{itemize}

\item Clases sociales:
\begin{enumerate}
	\item Patricios.
	\item Plebeyos.
	\item Clientes.
	\item Esclavos.
\end{enumerate}

\item Instituciones:
\begin{enumerate}
	\item Rey.
	\item Senado.
	\item Asamblea.
\end{enumerate}

\end{itemize}

\subsubsection{La República}

\begin{itemize}

\item Durante este periodo, patricios y plebeyos se igualan en derechos y obligaciones.

\item Los romanos ocupan toda la península.

\item Enfrentan a los cartagineses entre el 264 A.C. y el 146 A.C.

\item Ocupan el medio oriente.

\item La república colapso a raíz de las guerras civiles y se convirtió en un imperio.

\item Instituciones:
\begin{enumerate}
	\item Cónsules.
	\item Senado.
	\item Asamblea.
	\item Dictadura.
	\item Otros (edil, censor, pretor).
\end{enumerate}

\end{itemize}

\subsubsection{El Imperio}

\begin{itemize}

\item Durante este periodo, Roma se expandió enormemente.

\item Hacia el siglo III D.C. fue ocupada por los pueblos bárbaros que terminaron ocupando su capital occidental (Roma).

\item El lado oriental (Constantinopla) cayó en poder de los turcos en 1.453 D.C.

\end{itemize}

\subsubsection{Cultura Romana}

\begin{itemize}

\item Lengua: Latín.

\item Literatura:
\begin{itemize}
	\item Influencia griega.
	\item Géneros:
		\begin{itemize}
			\item \textbf{Épica:}
				\begin{itemize}
					\item Virgilio ``La Eneida''.
				\end{itemize}
			\item \textbf{Lírica:}
				\begin{itemize}
					\item Horacio: ``Carpe Diem''.
					\item Ovidio: ``Metamorfosis'', ``El Arte de Amar''.
				\end{itemize}
			\item \textbf{Satírica:}
				\begin{itemize}
					\item Petronio ``El Satiricón''.
				\end{itemize}
		\end{itemize}
\end{itemize}

\item Historia:
\begin{itemize}
	\item Julio César (``Las Galias'').
	\item Plutarco (``Vidas Paralelas'').
\end{itemize}

\item Oratoria:
\begin{itemize}
	\item Cicerón: Uno de los juristas más importantes del mundo. Trabajó en el \emph{derecho natural}.
\end{itemize}

\item Ciencias Naturales:
\begin{itemize}
	\item Plinio \emph{el viejo}: ``La Naturaleza de las Cosas''.
\end{itemize}

\item Medicina:
\begin{itemize}
	\item Aulo Celso: Muy bueno en traumatología.
\end{itemize}

\item Oratoria:
\begin{itemize}
	\item Cicerón: Uno de los juristas más importantes del mundo. Trabajó en el \emph{derecho natural}.
\end{itemize}

\item Arquitectura e Ingeniería:
\begin{itemize}
	\item Teatros.
	\item Vías.
	\item Arcos del Triunfo.
	\item Puentes.
	\item Acueductos.
\end{itemize}

\end{itemize}

\paragraph{Derecho Romano}

\begin{itemize}

\item Fuentes:
\begin{itemize}
	\item La costumbre.
	\item Leyes de los pretores.
\end{itemize}

\item Disposiciones:
\begin{itemize}
	\item Senadores.
	\item Emperadores.
\end{itemize}

\item La \emph{jurisprudencia}: Conjunto de fallas uniformes (en el mismo sentido) emitido por los juicios.

\end{itemize}