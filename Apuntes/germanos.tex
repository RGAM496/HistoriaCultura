\subsection{Los Germanos}

\begin{itemize}

\item Pueblo integrado por:
\begin{enumerate}
	\item Francos.
	\item Anglos.
	\item Sajones.
	\item Normandos.
	\item Vándalos (asolaban las ciudades donde pasaban).
	\item Godos.
	\begin{enumerate}
		\item Visigodos.
		\item Ostrogodos.
	\end{enumerate}
\end{enumerate}

\item Los romanos los llamaban \emph{bárbaros}, porque vivían fuera de las fronteras de Roma y no hablaban latín.

\item Ocuparon Roma en el 476 D.C. pero no destruyeron la cultura romana.

\item La autoridad absoluta era del padre.

\item El clan era responsable de la conducta de sus miembros.

\item Justicia:
\begin{enumerate}
	\item \emph{Venganza privada}: posibilidad de arreglo pecuniario (dinero).
	\item \emph{Las ordalías}: juicio de Dios.
	\item \emph{Duelo judicial}: solo para los nobles (el derrotado es el culpable).
	\item La tradición está por encima de la voluntad del soberano.
\end{enumerate}

\item Religión:
\begin{itemize}
	\item Para los anglos y los sajones.
	\begin{enumerate}
		\item \emph{Odín}: padre de los dioses, señor de la guerra.
		\item \emph{Tor}: dios de la espada.
		\item \emph{Walkirias}: diosas guerreras.
		\item \emph{Loki}: dios del infierno.
	\end{enumerate}
\end{itemize}

\end{itemize}