\subsection{El Islam}

\begin{itemize}

\item \emph{Islam}: someterse a la voluntad de Alá.

\item Fundador: Mahoma.

\item Libro sagrado: El Coran (recitación).

\item En el 622 D.C. huyó de La Meca a Yatrib (conocida también como Medina). A esta huída se la conoce como \emph{hégira}, y da inicio al calendario musulmán.

\item Doctrina:
\begin{enumerate}
	\item Un dios único, Alá, y Mahoma su profeta.
	\item Dar lismosnas.
	\item Guerra santa (\emph{yihad}).
	\item ayunar en el mes de \emph{Ramadán}.
	\item Orar cinco veces al dia.
	\item No consumir carne de cerdo ni beber alcohol.
	\item Peregrinar a La Meca.
\end{enumerate}

\item Fuentes:
\begin{itemize}
	\item Judaísmo.
	\begin{enumerate}
	\item Monoteísmo.
	\item Prohibición de la usura.
	\end{enumerate}
	\item Cristianismo.
	\begin{enumerate}
		\item Resurrección.
		\item Juicio final.
		\item Premios y castigos.
	\end{enumerate}
\end{itemize}

Sectas principales:
\begin{enumerate}
	\item \emph{Sunitas}: político-religiosos.
	\item \emph{Shiitas}: altos cargos para descendientes del profeta.
	\item \emph{Sufis}: místicos.
\end{enumerate}

\item Tras la muerte de Mahoma (632), ocuparon todo el norte de África.
A España llegaron en el 711.

\item Dividieron el imperio en tres califatos:
\begin{itemize}
	\item Bagdad.
	\item El Cairo.
	\item Córdoba.
\end{itemize}

\item En el siglo X, los turcos reemplazaron a los árabes en el control de la religión.
Fueron éstos quienes, en 1453, tomaron Constantinopla y pusieron fin al \emph{Imperio Romano de Oriente}.

\end{itemize}