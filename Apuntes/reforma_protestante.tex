\subsection{Reforma Protestante}

\begin{itemize}

\item Fue un movimiento religioso que se originó en Alemania en el siglo XVI y cuya consecuencia fue la división de la iglesia católica.

\item Causas:
\begin{enumerate}
	\item Ignorancia de los clérigos.
	\item Avidez de la curia romana.
	\item El nepotismo.
	\item La venta de indulgencias plenarias.
	\item Conflictos entre monjes y seglares.
	\item Se impugnaron (refutaron) los dogmas de la iglesia.
	\item El espíritu crítico  incorporado por el renacimiento.
\end{enumerate}

\subsubsection{EL Luteranismo}

\item Líder: Martín Lutero.
\item Origen: Alemania (1537).

\item Doctrina:
\begin{enumerate}
	\item Sólo dos sacramentos, el bautismo y la comunión.
	\item Salvación por la fe y el arrepentimiento.
	\item La Biblia es libremente interpretada.
	\item Supresión del culto a los santos y a la Virgen María.
	\item Eliminación de las imágenes.
\end{enumerate}

\subsubsection{El Calvinismo}

\item Líder: Juan Calvino.
\item Origen: Suiza.

\item Doctrina:
\begin{enumerate}
	\item Similar al luteranismo.
	\item Incorpora la idea de la predestinación.
\end{enumerate}

\subsubsection{El Anglicanismo}

\item Origen: Inglaterra.
\item Líder: Enrique VIII.

\item Doctrina:
\begin{enumerate}
	\item Toma ideas calvinistas y católicas.
	\item El rey es la máxima autoridad de la iglesia.
\end{enumerate}

\item Entre 1531 y 1555, católicos y protestantes se enfrentaron militarmente.
La guerra concluyó con la paz de Ausburgo (cada príncipe elegiría la religión de sus súbditos).

\end{itemize}